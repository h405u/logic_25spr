\homework{7}

\begin{exercise}{7.1}
  [Enderton, 练习~1, 2, 5, 第~17~页]
  在英文和指定的一阶语言之间进行翻译。
  \begin{enumerate}[label=(\alph*)]
    \item Any uninteresting number with the property that all smaller numbers are interesting certainly is interesting. ($\forall$, for all things; $N$, is a number; $I$, interesting; $<$, is less than; $0$, a constant symbol intended to denote zero.)
    \item There is no number such that no number is less than it. (The same language as in Item a.)
    \item $\forall x(Nx\rightarrow Ix\rightarrow\neg\forall y(Ny\rightarrow Iy\rightarrow\neg x<y))$. (The same language as in Item a. There exists a relatively concise translation.)
    \item (i) You can fool some of the people all of the time. (ii) You can fool all of the people some of the time. (iii) You can’t fool all of the people all of the time. ($\forall$, for all things; $P$, is a person; $T$ , is a time; $F\ x\ y$, you can fool $x$ at $y$. One or more of the above may be ambiguous, in which case you will need more than one translation.)\qedhere
  \end{enumerate}
\end{exercise}

\begin{enumerate}[label=(\alph*)]
  \item $\forall x(Nx\wedge\neg Ix\wedge\forall y (Ny\wedge y<x\rightarrow Iy)\rightarrow Ix).$
  \item $\neg\exists x(Nx\wedge\neg\exists y(Ny\wedge y<x)).$
  \item Any interesting number is less than some interesting number.
  \item (i) Here ambiguity is in that it either says that there are some (fixed) people you can fool all time, or says that at every moment there are (some, not fixed) people you can fool, i.e. either $\exists x(Px\wedge\forall y(Ty\to Fxy))$ or $\forall y(Ty\to\exists x(Px\wedge Fxy))$. (ii) Here ambiguity is in that it either says that you can fool each person at some time (times can be different for different people), or says that at some (fixed) time you can fool everyone (at that specific time): $\forall x(Px\to\exists y(Ty\wedge Fxy))$ or $\exists y(Ty\wedge\forall x(Px\to Fxy))$. (iii) $\neg\forall x(Px\to\forall y(Ty\to Fxy)).$
\end{enumerate}

\begin{exercise}{7.2}
  [Enderton, 练习~1, 第~129~页]
  对于项 $u$,令 $u_t^x$ 表示将项 $t$ 应用于 $u$ 中变元 $x$ 所得到的表达式。请在不使用任何“替换”或其同义词的情况下重新表述此定义。
\end{exercise}

对于项 $t$ 和变元 $x$,我们定义 $\sigma_{x\mapsto t}:V\rightarrow T$,其中 $V$ 是变元的集合,$T$ 是项的集合,且 $\sigma_{x\mapsto t}$ 除将 $x$ 映射到 $t$ 外,在其他情况下为恒等映射。然后递归地定义扩展映射 $\overline{\sigma_{x\mapsto t}}:T\rightarrow T$:
\begin{enumerate}
  \item 对于每个变元 $v$,有 $\overline{\sigma_{x\mapsto t}}(v)=\sigma_{x\mapsto t}(v)$。
  \item 对于每个常数符号 $c$,有 $\overline{\sigma_{x\mapsto t}}(c)=c$。
  \item 对于项 $t_1,\dots,t_n$ 及 $n$ 元函数符号 $f$,
        \[\overline{\sigma_{x\mapsto t}}(f(t_1,\dots,t_n))=f(\overline{\sigma_{x\mapsto t}}(t_1),\dots,\overline{\sigma_{x\mapsto t}}(t_n)).\]
\end{enumerate}

\begin{exercise}{7.3}
  [Enderton, 练习~9, 第~130~页]
  \begin{enumerate}[label=(\alph*)]
    \item 通过两个例子说明 $(\varphi_y^x)_x^y$ 一般情况下并不等于 $\varphi$,其中第一个例子展示 $x$ 在 $(\varphi_y^x)_x^y$ 中出现在 $\varphi$ 中未出现的位置,第二个例子展示 $x$ 在 $\varphi$ 中出现在 $(\varphi_y^x)_x^y$ 中未出现的位置。
    \item 证明重新应用引理(\textit{Re-replacement lemma}):如果 $y$ 不在 $\varphi$ 中出现,则 $x$ 在 $\varphi_y^x$ 中是可替换的,且有 $(\varphi_y^x)_x^y=\varphi$。\qedhere
  \end{enumerate}
\end{exercise}

\begin{enumerate}[label=(\alph*)]
  \item $\varphi=P\ y$($y$ 在 $\varphi$ 中自由出现)和 $\forall y\ P\ x$(不可替换)。
  \item
        对 $\varphi$ 进行归纳证明。

        情形 1:对于原子式 $\varphi=P\ t_1,\dots,t_n$,我们有 $x$ 在 $\varphi_y^x$ 中是可替换的,且
        \[
          (\varphi_y^x)_x^y = ((P\ t_1,\dots,t_n)_y^x)_x^y = P\ ((t_1)_y^x)_x^y,\dots,((t_n)_y^x)_x^y=\varphi.
        \]
        情形 2:在归纳假设成立的前提下,由公式构造操作 $\mathcal{E}_{\neg}$、$\mathcal{E}_{\rightarrow}$ 和 $\mathcal{Q}_i$ 的定义可知归纳步骤成立,其中 $v_i\neq x$ 且 $v_i\neq y$(因为 $y$ 不在 $\varphi$ 中出现)。
        \newline
        情形 3:$\varphi=\forall x\ \psi$。此时有 $(\forall x\ \psi)_y^x=\forall x\ \psi$,其中 $y$ 不出现(自由地,因此 $x$ 是可替换的)。因此
        \[
          (\varphi_x^y)_y^x=((\forall x\ \psi)_y^x)_x^y=(\forall x\ \psi)_y^x=\forall x\ \psi=\varphi.
        \]
\end{enumerate}
