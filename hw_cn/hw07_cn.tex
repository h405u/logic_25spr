\homework{7}

\begin{exercise}{7.1}
  [Enderton, ex.~1, 2, 5, p.~17]
  在英文和指定的一阶语言之间进行翻译.
  \begin{enumerate}[label=(\alph*)]
    \item Any uninteresting number with the property that all smaller numbers are interesting certainly is interesting. ($\forall$, for all things; $N$, is a number; $I$, interesting; $<$, is less than; $0$, a constant symbol intended to denote zero.)
    \item There is no number such that no number is less than it. (The same language as in Item a.)
    \item $\forall x(Nx\rightarrow Ix\rightarrow\neg\forall y(Ny\rightarrow Iy\rightarrow\neg x<y))$. (The same language as in Item a. There exists a relatively concise translation.)
    \item (i) You can fool some of the people all of the time. (ii) You can fool all of the people some of the time. (iii) You can't fool all of the people all of the time. ($\forall$, for all things; $P$, is a person; $T$, is a time; $F\ x\ y$, you can fool $x$ at $y$. One or more of the above may be ambiguous, in which case you will need more than one translation.)\qedhere
  \end{enumerate}
\end{exercise}

\begin{enumerate}[label=(\alph*)]
  \item $\forall x(Nx\wedge\neg Ix\wedge\forall y (Ny\wedge y<x\rightarrow Iy)\rightarrow Ix).$
  \item $\neg\exists x(Nx\wedge\neg\exists y(Ny\wedge y<x)).$
  \item Any interesting number is less than some interesting number.
  \item (i) Here ambiguity is in that it either says that there are some (fixed) people you can fool all time, or says that at every moment there are (some, not fixed) people you can fool, i.e. either $\exists x(Px\wedge\forall y(Ty\to Fxy))$ or $\forall y(Ty\to\exists x(Px\wedge Fxy))$. (ii) Here ambiguity is in that it either says that you can fool each person at some time (times can be different for different people), or says that at some (fixed) time you can fool everyone (at that specific time): $\forall x(Px\to\exists y(Ty\wedge Fxy))$ or $\exists y(Ty\wedge\forall x(Px\to Fxy))$. (iii) $\neg\forall x(Px\to\forall y(Ty\to Fxy)).$
\end{enumerate}

\begin{exercise}{7.2}
  [Enderton, ex.~1, p.~129]
  对于一个项 $u$, 记 $u_t^x$ 为将变量 $x$ 用项 $t$ 代换得到的表达式. 请在不使用单词 ``replace" 及其同义词的情况下重新表述此定义.
\end{exercise}

对于项 $t$ 和变量 $x$, 定义映射 $\sigma_{x\mapsto t}:V\rightarrow T$, 其中 $V$ 是变量集合, $T$ 是项集合, $\sigma_{x\mapsto t}$ 与恒等映射相同, 但将 $x$ 映到 $t$. 进一步递归地定义其扩张 $\overline{\sigma_{x\mapsto t}}:T\rightarrow T$:
\begin{enumerate}
  \item 对每个变量 $v$, 有 $\overline{\sigma_{x\mapsto t}}(v)=\sigma_{x\mapsto t}(v)$.
  \item 对每个常量符号 $c$, 有 $\overline{\sigma_{x\mapsto t}}(c)=c$.
  \item 若 $t_1,\dots,t_n$ 为项且 $f$ 为 $n$ 元函数符号, 则
        \[\overline{\sigma_{x\mapsto t}}(f(t_1,\dots,t_n))=f(\overline{\sigma_{x\mapsto t}}(t_1),\dots,\overline{\sigma_{x\mapsto t}}(t_n)).\]
\end{enumerate}

\begin{exercise}{7.3}
  [Enderton, ex.~9, p.~130]
  \begin{enumerate}[label=(\alph*)]
    \item 请给出两个例子证明, $(\varphi_y^x)_x^y$ 一般并不等于 $\varphi$. 第一个例子应说明 $x$ 可能出现在 $(\varphi_y^x)_x^y$ 中而不出现在 $\varphi$ 的对应位置, 第二个例子应说明 $x$ 可能出现在 $\varphi$ 中而不出现在 $(\varphi_y^x)_x^y$ 的对应位置.
    \item 证明 \textit{Re-replacement lemma}: 若 $y$ 不出现在 $\varphi$ 中, 则 $x$ 对 $\varphi_y^x$ 中的 $y$ 是可替换的, 并且 $(\varphi_y^x)_x^y=\varphi$.\qedhere
  \end{enumerate}
\end{exercise}

\begin{enumerate}[label=(\alph*)]
  \item $\varphi=P\ y$ ($y$ 在 $\varphi$ 中自由出现) 与 $\varphi=\forall y\ P\ x$ ($y$ 对 $\forall y\ P\ x$ 中的 $x$ 不可替换).
  \item
        我们对 $\varphi$ 进行归纳证明.

        Case 1: 若 $\varphi$ 为原子式 $P\ t_1,\dots,t_n$, 则 $x$ 对 $\varphi_y^x$ 中的 $y$ 是可替换的, 且
        \[
          (\varphi_y^x)_x^y = ((P\ t_1,\dots,t_n)_y^x)_x^y = P\ ((t_1)_y^x)_x^y,\dots,((t_n)_y^x)_x^y=\varphi.
        \]
        Case 2: 设归纳假设成立, 则对于公式构造算子 $\mathcal{E}_{\neg}$, $\mathcal{E}_{\rightarrow}$ 和 $\mathcal{Q}_i$ (其中 $v_i\neq x$ 且 $v_i\neq y$, 因为 $y$ 不出现在 $\varphi$ 中), 归纳步骤由定义立即成立.\newline
        Case 3: 若 $\varphi=\forall x\ \psi$, 则 $(\forall x\ \psi)_y^x=\forall x\ \psi$, 其中 $y$ 不自由出现, 因而 $x$ 对 $(\forall x \psi)_y^x$ 中的 $y$ 是可替换的. 因此
        \[
          (\varphi_x^y)_y^x=((\forall x\ \psi)_y^x)_x^y=(\forall x\ \psi)_y^x=\forall x\ \psi=\varphi.
        \]
\end{enumerate}
