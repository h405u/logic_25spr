\homework{11}

\begin{exercise}{11.1}
  [Enderton, ex.~4, p.~146]
  设 $\Gamma=\{\neg\forall v_1 P v_1, Pv_2, Pv_3,\dots\}$. 问: $\Gamma$ 是否一致? $\Gamma$ 是否可满足?
\end{exercise}

它既一致又可满足. 令 $P^{\mathfrak{A}}=\{v_2,v_3,\dots\}$, 取 $s:V\rightarrow|\mathfrak{A}|$ 为恒等映射, 则对所有 $\gamma\in\Gamma$ 有 $\vDash_{\mathfrak{A}}\gamma[s]$.

\begin{exercise}{11.2}
  [Enderton, ex.~7, p.~146]
  对下列每个句子, 或给出推导, 或给出反例结构 (即使其为假的结构).
  \begin{enumerate}[label=(\alph*)]
    \item $\forall x (Qx \rightarrow \forall y \, Qy)$
    \item $(\exists x \, Px \rightarrow \forall y \, Qy) \rightarrow \forall z (Pz \rightarrow Qz)$
    \item $\forall z (Pz \rightarrow Qz) \rightarrow (\exists x \, Px \rightarrow \forall y \, Qy)$
    \item $\neg \exists y \, \forall x (Pxy \leftrightarrow \neg Pxx)$\qedhere
  \end{enumerate}
\end{exercise}

\begin{enumerate}[label=(\alph*)]
  \item 不成立. 取 $|\mathfrak A|=\{0,1\}$ 且 $Q^{\mathfrak A}=\{1\}$. 则 $\mathfrak A$ 为反例结构.
  \item 可证明, 推导如下:\newline
        $\begin{aligned}[t]
                       & \vdash(\exists xPx\to\forall yQy)\to \forall z(Pz\to Qz)                             \\
            \Leftarrow & \exists xPx\to\forall yQy\vdash \forall z(Pz\to Qz)      & \text{by ded,}            \\
            \Leftarrow & \{\exists xPx\to\forall yQy, Pz\}\vdash Qz               & \text{by gen and ded,}    \\
                       & \text{which we show directly:}                                                       \\
            1.         & \forall x\neg Px\vdash \neg Pz                           & \text{Ax.2; ded.}         \\
            2.         & Pz\vdash \neg\forall x\neg Px                            & \text{1; contraposition.} \\
            3.         & \{\exists xPx\to\forall yQy, Pz\}\vdash \forall yQy      & \text{2; MP.}             \\
            4.         & \vdash\forall yQy\to Qz                                  & \text{Ax.2.}              \\
            5.         & \{\exists xPx\to\forall yQy, Pz\}\vdash Qz               & \text{3; 4; MP.}          \\
          \end{aligned}$
  \item 不成立. 取 $|\mathfrak A|=\{0,1\}$ 且 $P^{\mathfrak A}=Q^{\mathfrak A}=\{1\}$. 则 $\mathfrak A$ 为反例结构.
  \item 可证明, 推导如下:\newline
        $\begin{aligned}[t]
                       & \vdash\neg\exists y\forall x(Pxy \leftrightarrow \neg Pxx)                                                          \\
            \Leftarrow & \vdash\forall y\neg \forall x(Pxy \leftrightarrow\neg Pxx)                                 & \text{by Ax.1 and MP,} \\
            \Leftarrow & \vdash\neg \forall x(Pxy \leftrightarrow\neg Pxx)                                          & \text{by gen,}         \\
                       & \text{which we show directly:}                                                                                      \\
            1.         & \vdash\neg(Pxy \leftrightarrow \neg Pxx)_y^x                                               & \text{Ax.1.}           \\
            2.         & \vdash\forall x(Pxy \leftrightarrow \neg Pxx)\to (Pxy \leftrightarrow \neg Pxx)_y^x        & \text{Ax.2.}           \\
            3.         & \vdash\neg(Pxy \leftrightarrow \neg Pxx)_y^x\to\neg\forall x(Pxy \leftrightarrow \neg Pxx) & \text{2; Ax.1; MP.}    \\
            4.         & \vdash\neg \forall x(Pxy \leftrightarrow\neg Pxx)                                          & \text{1; 3; MP.}       \\
          \end{aligned}$
\end{enumerate}

\begin{exercise}{11.3}
  [Enderton, ex.~8, p.~146]
  设语言 (含相等) 仅有量词和二元谓词符号 $P$. 取结构 $\mathfrak{A}$, 其中 $|\mathfrak{A}|=\mathbb{Z}$, 且 $\langle a,b\rangle\in P^{\mathfrak{A}}$ 当且仅当 $|a-b|=1$. 因此 $\mathfrak{A}$ 如同一条无限链:
  \[
    \cdots \longleftrightarrow \bullet \longleftrightarrow \bullet \longleftrightarrow \bullet \longleftrightarrow \cdots
  \]
  证明: 存在与 $\mathfrak{A}$ 初等等价但非连通的结构 $\mathfrak{B}$. (连通指对 $|\mathfrak{B}|$ 中任意两点存在路径相连. 长度为 $n$ 的路径 $\langle p_0,p_1,\dots,p_n\rangle$ 满足 $p_0=a$, $p_n=b$, 且对每个 $i$ 有 $\langle p_i,p_{i+1}\rangle\in P^{\mathfrak{B}}$.) 建议: 加入常量符号 $c,d$, 写句子声明 $c,d$ 距离很远, 再用紧致性.
\end{exercise}

在语言中增添常量 $c,d$. 对每个整数 $k\ge 0$, 构造句子 $\lambda_k$ 表示 “$c$ 与 $d$ 的距离不是 $k$”. 例如,
\begin{align*}
  \lambda_0 & =\neg c=d,                                               \\
  \lambda_1 & =\forall p_1(Pcp_1\to\neg p_1=d),                        \\
  \lambda_2 & =\forall p_1\forall p_2(Pcp_1\to Pp_1p_2\to \neg p_2=d).
\end{align*}
设 $\Sigma=\{\lambda_0,\lambda_1,\lambda_2,\dots\}$. 取 $\Sigma\cup\mathrm{Th}\mathfrak{A}$ 的任意有限子集, 它在某个 $\mathfrak{A}_k$ 中成立, 其中 $|c^{\mathfrak{A}_k}-d^{\mathfrak{A}_k}|>k$. 故由紧致性, $\Sigma\cup\mathrm{Th}\mathfrak{A}$ 有模型
\[
  \mathfrak{B}=(|\mathfrak{B}|;P^{\mathfrak{B}},=^{\mathfrak{B}},c^{\mathfrak{B}},d^{\mathfrak{B}}).
\]
记 $\mathfrak{B}_0$ 为去掉新常量后的结构: $\mathfrak{B}_0=(|\mathfrak{B}|,P^{\mathfrak{B}},=^{\mathfrak{B}})$. 则 $\mathfrak{B}_0\models\mathrm{Th}\mathfrak{A}$, 因而 $\mathfrak{B}_0\equiv\mathfrak{A}$. 然而 $c^{\mathfrak{B}}$ 与 $d^{\mathfrak{B}}$ 间不存在路径, 故 $\mathfrak{B}_0$ 非连通.

注. 可将 $c,d$ 各自所在的无限链与 $\mathbb{Z}$ 看作三条彼此不连通的链, 它们共同组成 $|\mathfrak{B}|$. 对 $\mathfrak{B}_0$ 的任何有限子结构, 看起来都与 $\mathfrak{A}$ 的某段有限子链相同.

\begin{exercise}{11.4}
  [Enderton, ex.~11, p.~100]
  在结构 $(\mathbb{N}; +,\cdot)$ 中 (语言含相等及符号 $\forall,+,\cdot$), 对下列关系给出定义它们的公式.
  \begin{enumerate}
    \item $\{0\}$.
    \item $\{1\}$.
    \item $\{\langle m,n\rangle\mid n\text{ 是 } m\text{ 的后继}\}$.
    \item $\{\langle m,n\rangle\mid m<n\}$.\qedhere
  \end{enumerate}
\end{exercise}

\begin{enumerate}
  \item $\forall x\ x+a=x$.
  \item $\forall x\ x\cdot a=x$.
  \item $\exists y\forall x(x\cdot y=x\wedge n=m+y)$.
  \item $\exists y\forall x\exists k(x+y=x\wedge \neg k=y\wedge n=m+k)$.
\end{enumerate}

\begin{exercise}{11.5}
  [Enderton, ex.~6, p.~146]
  设 $\Sigma_1,\Sigma_2$ 为句子集, 且不存在同时满足两者的结构. 证明存在句子 $\tau$ 使得
  \[
    \text{Mod }\Sigma_1\subseteq\text{Mod }\tau,\quad \text{且}\quad \text{Mod }\Sigma_2\subseteq\text{Mod }\lnot\tau.
  \]
  (等价表述: 不相交的 $\text{EC}_\Delta$ 类可以用一条 EC 类分开.) 提示: $\Sigma_1\cup\Sigma_2$ 不可满足, 用紧致性.
\end{exercise}

可设 $\Sigma_1,\Sigma_2$ 均可满足 (其余情形显然). 因 $\Sigma_1\cup\Sigma_2$ 不可满足, 故其某有限子集 $\Sigma_0$ 不可满足. 设 $\alpha$ 为 $\Sigma_0\cap\Sigma_1$ 中句子的合取. 则 $\Sigma_1\vdash\alpha$. 若 $\Sigma_2\not\vdash\neg\alpha$, 则 $\Sigma_2;\alpha$ 可满足, 这与 $\Sigma_0$ 不可满足矛盾. 因而 $\Sigma_2\vdash\neg\alpha$. 取 $\tau=\alpha$, 则满足要求.
