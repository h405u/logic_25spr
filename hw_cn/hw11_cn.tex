\homework{11}

\begin{exercise}{11.1}
  [Enderton,第 4 题,第 146 页]
  设 $\Gamma=\{\neg\forall v_1 P v_1, Pv_2, Pv_3,\dots\}$。$\Gamma$ 是一致的吗?$\Gamma$ 是可满足的吗?
\end{exercise}

\textcolor{red}{to do}

\begin{exercise}{11.2}
  [Enderton,第 7 题,第 146 页]
  对于下列每个句子,要么说明其存在一个推导,要么给出一个反模型(即在其中该句子为假的结构)。
  \begin{enumerate}[label=(\alph*)]
    \item $\forall x (Qx \rightarrow \forall y \, Qy)$
    \item $(\exists x \, Px \rightarrow \forall y \, Qy) \rightarrow \forall z (Pz \rightarrow Qz)$
    \item $\forall z (Pz \rightarrow Qz) \rightarrow (\exists x \, Px \rightarrow \forall y \, Qy)$
    \item $\neg \exists y \, \forall x (Pxy \leftrightarrow \neg Pxx)$\qedhere
  \end{enumerate}
\end{exercise}

\textcolor{red}{to do}

\begin{exercise}{11.3}
  [Enderton,第 8 题,第 146 页]
  假设该语言(带等号)仅包含量词 $\forall$ 和谓词符号 $P$,其中 $P$ 是一个二元谓词符号。设 $\mathfrak{A}$ 是一个结构,其 $|\mathfrak{A}| = \mathbb{Z}$,即整数集合(正整数、负整数和零),并且当且仅当 $|a - b| = 1$ 时,有 $\langle a, b \rangle \in P^{\mathfrak{A}}$。因此,$\mathfrak{A}$ 看起来像一个无限图:
  \[
    \cdots \longleftrightarrow \bullet \longleftrightarrow \bullet \longleftrightarrow \bullet \longleftrightarrow \cdots
  \]

  证明存在一个与其初等等价的结构 $\mathfrak{B}$,但它不是连通的。(连通,是指对于 $|\mathfrak{B}|$ 中任意两个元素,存在一条路径连接它们。一条从 $a$ 到 $b$ 的长度为 $n$的路径是一个序列 $\langle p_0, p_1, \ldots, p_n \rangle$,其中 $a = p_0$ 且 $b = p_n$,并且对于每个 $i$,都有 $\langle p_i, p_{i+1} \rangle \in P^{\mathfrak{B}}$。)提示:加入常元符号 $c$ 和 $d$。写出表示 $c$ 和 $d$ 距离很远的句子。应用紧致性。
\end{exercise}

\textcolor{red}{to do}

\begin{exercise}{11.4}
  [Enderton,第 11 题,第 100 页]
  对于下列每个关系,在 \((\mathbb{N}; +, \cdot)\) 中给出一个定义它的公式。(语言被假设包含等号以及参数 \(\forall\)、\(+\)、和 \(\cdot\)。)
  \begin{enumerate}
    \item \(\{0\}\).
    \item \(\{1\}\).
    \item \(\{ \langle m, n \rangle \mid n \text{ 是 } m \text{ 在 } \mathbb{N} \text{ 中的后继} \}\).
    \item \(\{ \langle m, n \rangle \mid m < n \text{ 在 } \mathbb{N} \text{ 中} \}\).\qedhere
  \end{enumerate}
\end{exercise}

\textcolor{red}{
  Assuming Definability
}

\begin{exercise}{11.5}
  [Enderton,第 6 题,第 146 页]
  设 $\Sigma_1$ 和 $\Sigma_2$ 是两个句子的集合,且没有模型同时满足 $\Sigma_1$ 和 $\Sigma_2$。证明存在一个句子 $\tau$,使得
  \[
    \text{Mod } \Sigma_1 \subseteq \text{Mod } \tau \quad \text{且} \quad \text{Mod } \Sigma_2 \subseteq \text{Mod } \lnot \tau.
  \]
  (这可以表述为:不相交的 $\text{EC}_\Delta$ 类可以被某个 EC 类区分开。)提示:$\Sigma_1 \cup \Sigma_2$ 是不可满足的;应用紧致性。
\end{exercise}

\textcolor{red}{Assuming Mod}
