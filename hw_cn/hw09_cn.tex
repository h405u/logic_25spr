\homework{9}

\begin{exercise}{9.1}
  [Enderton, 练习 1,第 99 页]
  证明:(a) $\Gamma;\alpha\vDash \varphi$ 当且仅当 $\Gamma\vDash(\alpha\to \varphi)$;(b) $\varphi\vDash\Dashv \psi$ 当且仅当 $\vDash(\varphi \leftrightarrow \psi)$。
\end{exercise}

\begin{exercise}{9.2}
  [Enderton, 练习 4,第 99 页]
  证明:如果 $x$ 不在 $\alpha$ 中自由出现,则 $\alpha\vDash\forall x \alpha$。
\end{exercise}

\begin{exercise}{9.3}
  [Enderton, 练习 6,第 99 页]
  证明:公式 $\theta$ 是有效的,当且仅当 $\forall\theta$ 是有效的。
\end{exercise}

\begin{exercise}{9.4}
  [Enderton, 练习 7,第 99 页]
  将“$\mathfrak{A}$ 在赋值 $s$ 下满足 $\varphi$”的定义,重新表述为递归地定义一个函数 $\overline{h}$,使得 $\mathfrak{A}$ 在赋值 $s$ 下满足 $\varphi$ 当且仅当 $s\in \overline{h}(\varphi)$。
\end{exercise}

\begin{exercise}{9.5}
  [Enderton, 练习 9,第 100 页]
  假设语言中含有等号和一个二元谓词符号 $P$。对于下列每一个条件,找出一个句子 $\sigma$,使得结构 $\mathfrak{A}$ 是 $\sigma$ 的一个模型,当且仅当该条件成立。
  \begin{enumerate}[label=(\alph*)]
    \item $|\mathfrak{A}|$ 恰好有两个元素。
    \item $P^{\mathfrak{A}}$ 是从 $|\mathfrak{A}|$ 到 $|\mathfrak{A}|$ 的函数。(函数指单值关系。如第 0 章所述,若 $f$ 是从 $A$ 到 $B$ 的函数,则 $f$ 的定义域是 $A$,而值域是 $B$ 的子集,不一定是适当子集。)
    \item $P^{\mathfrak{A}}$ 是 $|\mathfrak{A}|$ 上的置换,即 $P^{\mathfrak{A}}$ 是定义域和值域均为 $|\mathfrak{A}|$ 的双射。\qedhere
  \end{enumerate}
\end{exercise}

% \begin{exercise}{9.6}
%   [Enderton, 练习 10,第 100 页]
%   证明
%   \[
%     \models_{\mathfrak{A}} \forall v_2 \, Q v_1 v_2 [[c^\mathfrak{A}]] \quad \text{iff} \quad \models_{\mathfrak{A}} \forall v_2 \, Q c v_2
%   \]
%   其中 $Q$ 是二元谓词符号,$c$ 是常数符号。
% \end{exercise}
%
% \begin{exercise}{9.7}
%   [Enderton, 练习 11,第 100 页]
%   对于下列每一个关系,给出在 $(\mathbb{N}; +, \cdot)$ 中定义它的公式。(语言中包含等号以及符号 $\forall$、$+$ 和 $\cdot$。)
%   \begin{enumerate}
%     \item $\{0\}$。
%     \item $\{1\}$。
%     \item $\{\langle m, n \rangle \mid n$ 是 $m$ 在 $\mathbb{N}$ 中的后继元素$\}$。
%     \item $\{\langle m, n \rangle \mid m<n$ 在 $\mathbb{N}$ 中成立$\}$。\qedhere
%   \end{enumerate}
% \end{exercise}
