\homework{9}

\begin{exercise}{9.1}
  [Enderton, ex.~1, p.~99]
  证明: (a) $\Gamma;\alpha\vDash \varphi$ 当且仅当 $\Gamma\vDash(\alpha\to \varphi)$, (b) $\varphi\vDash\Dashv \psi$ 当且仅当 $\vDash(\varphi \leftrightarrow \psi)$.
\end{exercise}

\begin{enumerate}[label=(\alph*)]
  \item
        $\begin{aligned}[t]
            \Gamma;\alpha\vDash \varphi & \Leftrightarrow(\forall\tau\ \tau\in\Gamma;\alpha\rightarrow\ \vDash_{\mathfrak{A}}\tau[s])\rightarrow\ \vDash_{\mathfrak{A}}\varphi[s]                                 \\
                                        & \Leftrightarrow(\forall\tau\ \tau\in \Gamma\rightarrow\ \vDash_{\mathfrak{A}}\tau[s])\ \wedge\vDash_{\mathfrak{A}}\alpha[s]\rightarrow\ \vDash_{\mathfrak{A}}\varphi[s] \\
                                        & \Leftrightarrow(\forall\tau\ \tau\in \Gamma\rightarrow\ \vDash_{\mathfrak{A}}\tau[s])\rightarrow\ \vDash_{\mathfrak{A}}(\alpha\rightarrow \varphi)[s]                   \\
                                        & \Leftrightarrow \Gamma\vDash (\alpha\rightarrow \varphi).
          \end{aligned}$
  \item 在 (a) 中令 $\Gamma=\emptyset$, 得 $\varphi\vDash \psi$ 当且仅当 $\vDash(\varphi\to \psi)$, 同理 $\psi\vDash \varphi$ 当且仅当 $\vDash(\psi\to \varphi)$, 因此命题得证.
\end{enumerate}

注意: 本证明中出现的 $\wedge$, $\Leftrightarrow$, $\Rightarrow$ 只是元推理的简写.

\begin{exercise}{9.2}
  证明: 若 $x$ 在 $\alpha$ 中不自由出现, 则 $\alpha\vDash\forall x \alpha$.
\end{exercise}

固定 $\mathfrak{A}$ 与 $s$. 对每个 $d\in|\mathfrak{A}|$, 赋值 $s$ 与 $s(x|d)$ 在 $\alpha$ 的所有自由变量上一致, 由课件定理 2.17 可得 $\vDash_{\mathfrak{A}}\alpha[s]\Leftrightarrow\vDash_{\mathfrak{A}}\alpha[s(x|d)]$. 于是 $\vDash_{\mathfrak{A}}\alpha[s]$ 等价于 $\vDash_{\mathfrak{A}}\forall x\ \alpha[s]$, 故结论成立.

\begin{exercise}{9.3}
  证明: 公式 $\theta$ 有效当且仅当 $\forall x\theta$ 有效.
\end{exercise}

固定 $\mathfrak{A}$ 与 $s$. 若 $\theta$ 有效, 则对每个 $d\in|\mathfrak{A}|$ 有 $\vDash_{\mathfrak{A}}\theta[s(x|d)]$, 这正是 $\vDash_{\mathfrak{A}}\forall x\ \theta[s]$. 反过来, 取 $d=s(x)$ 使得 $s(x|d)=s$, 则由 $\vDash_{\mathfrak{A}}\forall x\ \theta[s]$ 推得 $\vDash_{\mathfrak{A}}\theta[s]$. 因此 $\theta$ 有效当且仅当 $\forall x\theta$ 有效.

\begin{exercise}{9.4}
  通过递归地定义函数 $\overline{h}$, 重新表述 “$\mathfrak{A}$ 在赋值 $s$ 下满足 $\varphi$”, 使得 $\mathfrak{A}$ 在赋值 $s$ 下满足 $\varphi$ 当且仅当 $s\in\overline{h}(\varphi)$.
\end{exercise}

首先定义 $h:A\rightarrow \mathcal{P}(|\mathfrak{A}|^V)$, 其中 $A$ 为原子公式的集合, $V$ 是变量集合, $\mathcal{P}(|\mathfrak{A}|^V)$ 为 $V$ 到 $|\mathfrak A|$ 的映射组成的集合的子集的集合. 对于 $n$ 元谓词参数 $P$ (若语言含有 $=$, 视其为 $2$ 元谓词),
\[
  h(P\ t_1\cdots t_n)=\{s:V\rightarrow|\mathfrak{A}|\mid\langle\bar{s}(t_1),\dots,\bar{s}(t_n)\rangle\in P^{\mathfrak{A}}\}.
\]

然后把 $h$ 扩张为以所有公式为定义域的 $\bar{h}$:
\begin{enumerate}
  \item 对原子式有 $h(\varphi)\subseteq\bar{h}(\varphi)$.
  \item $\bar{h}(\neg \varphi)=\{s:V\rightarrow|\mathfrak{A}|\mid s\notin\bar{h}(\varphi)\}$.
  \item $\bar{h}(\varphi\rightarrow \psi)=\bar{h}(\varphi)\cup\bar{h}(\psi)$.
  \item $\bar{h}(\forall x\ \varphi)=\{s:V\rightarrow|\mathfrak{A}|\mid\text{对于每个 }d\in|\mathfrak{A}|, s(x|d)\in\bar{h}(\varphi)\}$.
\end{enumerate}
最终定义
\[
  \vDash_{\mathfrak{A}}\varphi[s]\text{ iff }s\in\bar{h}(\varphi).\qedhere
\]

\begin{exercise}{9.5}
  [Enderton, ex.~9, p.~100]
  设语言含有相等符号及二元谓词符号 $P$. 对下列每一条件, 构造句子 $\sigma$, 使得结构 $\mathfrak{A}$ 成为 $\sigma$ 的模型当且仅当该条件成立.
  \begin{enumerate}[label=(\alph*)]
    \item $|\mathfrak{A}|$ 恰有两个元素.
    \item $P^{\mathfrak{A}}$ 是从 $|\mathfrak{A}|$ 映入 $|\mathfrak{A}|$ 的函数. (函数指单值关系. 若 $f$ 是从 $A$ 到 $B$ 的函数, 则 $\operatorname{dom}f=A$, 而 $\operatorname{rng}f\subseteq B$.)
    \item $P^{\mathfrak{A}}$ 是 $|\mathfrak{A}|$ 的置换, 即 $P^{\mathfrak{A}}$ 是定义域和值域均为 $|\mathfrak{A}|$ 的一一对应函数.\qedhere
  \end{enumerate}
\end{exercise}

\begin{enumerate}
  \item $\exists a\exists b\forall c(\neg a=b\wedge(c=a\vee c=b))$.
  \item $\forall x\exists y\forall z(P\ xy\wedge(P\ xz\to y=z))$.
  \item $\forall x\exists y\forall z\exists p\forall q\forall r(P\ xy\wedge(P\ xz\to y=z)\wedge P\ pq\wedge(P\ rq\to p=r))$.
\end{enumerate}
