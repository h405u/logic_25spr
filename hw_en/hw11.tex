\homework{11}

\begin{exercise}{11.1}
  [Enderton, ex.~4, p.~146]
  Let $\Gamma=\{\neg\forall v_1 P v_1, Pv_2, Pv_3,\dots\}.$ Is $\Gamma$ consistent? Is $\Gamma$ satisfiable?
\end{exercise}

It is consistent and satisfiable. Define $P^{\mathfrak{A}}=\{v_2,v_3,\dots\}$ and $s:V\rightarrow|\mathfrak{A}|$ as identity, it follows that for all $\gamma\in \Gamma$, $\vDash_{\mathfrak{A}}\gamma[s]$.

\begin{exercise}{11.2}
  [Enderton, ex.~7, p.~146]
  For each of the following sentences, either show there is a deduction or give a counter-model (i.e., a structure in which it is false.)
  \begin{enumerate}[label=(\alph*)]
    \item $\forall x (Qx \rightarrow \forall y \, Qy)$
    \item $(\exists x \, Px \rightarrow \forall y \, Qy) \rightarrow \forall z (Pz \rightarrow Qz)$
    \item $\forall z (Pz \rightarrow Qz) \rightarrow (\exists x \, Px \rightarrow \forall y \, Qy)$
    \item $\neg \exists y \, \forall x (Pxy \leftrightarrow \neg Pxx)$\qedhere
  \end{enumerate}
\end{exercise}

\begin{enumerate}[label=(\alph*)]
  \item Not valid. Let $|\mathfrak A|=\{0,1\}$ and $Q^{\mathfrak A}=\{1\}$. Then $\mathfrak A$ is a counter-model.
  \item
        $\begin{aligned}[t]
                       & \vdash(\exists xPx\to\forall yQy)\to \forall z(Pz\to Qz)                             \\
            \Leftarrow & \exists xPx\to\forall yQy\vdash \forall z(Pz\to Qz)      & \text{by ded,}            \\
            \Leftarrow & \{\exists xPx\to\forall yQy, Pz\}\vdash Qz               & \text{by gen and ded,}    \\
                       & \text{which we show directly:}                                                       \\
            1.         & \forall x\neg Px\vdash \neg Pz                           & \text{Ax.2; ded.}         \\
            2.         & Pz\vdash \neg\forall x\neg Px                            & \text{1; contraposition.} \\
            3.         & \{\exists xPx\to\forall yQy, Pz\}\vdash \forall yQy      & \text{2; MP.}             \\
            4.         & \vdash\forall yQy\to Qz                                  & \text{Ax.2.}              \\
            5.         & \{\exists xPx\to\forall yQy, Pz\}\vdash Qz               & \text{3; 4; MP.}          \\
          \end{aligned}$
  \item Not valid. Let $|\mathfrak A|=\{0,1\}$ and $P^{\mathfrak A}=Q^{\mathfrak A}=\{1\}$. Then $\mathfrak A$ is a counter-model.
  \item
        $\begin{aligned}[t]
                       & \vdash\neg\exists y\forall x(Pxy \leftrightarrow \neg Pxx)                                                          \\
            \Leftarrow & \vdash\forall y\neg \forall x(Pxy \leftrightarrow\neg Pxx)                                 & \text{by Ax.1 and MP,} \\
            \Leftarrow & \vdash\neg \forall x(Pxy \leftrightarrow\neg Pxx)                                          & \text{by gen,}         \\
                       & \text{which we show directly:}                                                                                      \\
            1.         & \vdash\neg(Pxy \leftrightarrow \neg Pxx)_y^x                                               & \text{Ax.1.}           \\
            2.         & \vdash\forall x(Pxy \leftrightarrow \neg Pxx)\to (Pxy \leftrightarrow \neg Pxx)_y^x        & \text{Ax.2.}           \\
            3.         & \vdash\neg(Pxy \leftrightarrow \neg Pxx)_y^x\to\neg\forall x(Pxy \leftrightarrow \neg Pxx) & \text{2; Ax.1; MP.}    \\
            4.         & \vdash\neg \forall x(Pxy \leftrightarrow\neg Pxx)                                          & \text{1; 3; MP.}       \\
          \end{aligned}$
\end{enumerate}

\begin{exercise}{11.3}
  [Enderton, ex.~8, p.~146]
  Assume the language (with equality) has just the parameters $\forall$ and $P$, where $P$ is a two-place predicate symbol. Let $\mathfrak{A}$ be the structure with $|\mathfrak{A}| = \mathbb{Z}$, the set of integers (positive, negative, and zero), and with $\langle a, b \rangle \in P^{\mathfrak{A}}$ iff $|a - b| = 1$. Thus $\mathfrak{A}$ looks like an infinite graph:
  \[
    \cdots \longleftrightarrow \bullet \longleftrightarrow \bullet \longleftrightarrow \bullet \longleftrightarrow \cdots
  \]

  Show that there is an elementarily equivalent structure $\mathfrak{B}$ that is not connected. (Being \emph{connected} means that for every two members of $|\mathfrak{B}|$, there is a path between them. A \emph{path} — of length $n$ — from $a$ to $b$ is a sequence $\langle p_0, p_1, \ldots, p_n \rangle$ with $a = p_0$ and $b = p_n$ and $\langle p_i, p_{i+1} \rangle \in P^{\mathfrak{B}}$ for each $i$.) \textit{Suggestion}: Add constant symbols $c$ and $d$. Write down sentences saying $c$ and $d$ are far apart. Apply compactness.
\end{exercise}

Expand the language by adding two new constant symbols $c$ and $d$. For each integer $k\geq 0$, we can find a sentence $\lambda_k$ that translates, ``The distance between $c$ and $d$ is \textit{not} $k$.'' For example,
\begin{align*}
  \lambda_0 & =\neg c=d                                                \\
  \lambda_1 & =\forall p_1(Pcp_1\to\neg p_1=d),                        \\
  \lambda_2 & =\forall p_1\forall p_2(Pcp_1\to Pp_1p_2\to \neg p_2=d).
\end{align*}
Let $\Sigma=\{\lambda_0,\lambda_1,\lambda_2,\dots\}.$ Consider a finite subset of $\Sigma\cup \mathrm{Th}\mathfrak{A}$. That subset is true in $\mathfrak{A}_k$ such that $|c^{\mathfrak{A}_k}-d^{\mathfrak{A}_k}|>k$ for some large $k$. So by compactness $\Sigma\cup \mathrm{Th}\mathfrak{A}$ has a model
\[
  \mathfrak{B}=(|\mathfrak{B}|;\mathrm{P}^{\mathfrak{B}},=^{\mathfrak{B}},c^{\mathfrak{B}},d^{\mathfrak{B}})
\]
Let $\mathfrak{B}_0$ be the restriction of $\mathfrak{B}$ to the original language: $\mathfrak{B}_0=(|\mathfrak{B}|,\mathrm{P}^{\mathfrak{B}},=^{\mathfrak{B}})$. We have that $\mathfrak B_0$ is a model of $\mathrm{Th}\mathfrak A$, so $\mathfrak{B}_0\equiv \mathfrak{A}$ (you may try to prove this). Note that $c^{\mathfrak{B}}\in|\mathfrak{B}|$ and $d^{\mathfrak{B}}\in|\mathfrak{B}|$, but there is no path between them.

\textit{Comment.} One might ask: every member of the universe should be connected to two unique nodes, then to which nodes is $c^{\mathfrak{B}}$ connected? Well, consider not $c$ and $d$ are located on the single infinite graph that $\mathfrak{A}$ indicates, but that $c$ and $d$ and $\mathbb{Z}$ are on 3 seperate infinite graphs, which together consititute our construction of $|\mathfrak{B}|$. That should make a better (possible) interpretation of what we have been effectively doing. One may feel that $c$ and $d$ are far apart but connected, but that is not the case in $\mathfrak{B}$. All we have is that every finite piece of $\mathfrak{B}_0$ looks like a finite segment of $\mathfrak{A}$.

\begin{exercise}{11.4}
  [Enderton, ex.~11, p.~100]
  For each of the following relations, give a formula which defines it in \((\mathbb{N}; +, \cdot)\). (The language is assumed to have equality and the parameters \(\forall\), \(+\), and \(\cdot\)).
  \begin{enumerate}
    \item \(\{0\}\).
    \item \(\{1\}\).
    \item \(\{ \langle m, n \rangle \mid n \text{ is the successor of } m \text{ in } \mathbb{N} \}\).
    \item \(\{ \langle m, n \rangle \mid m < n \text{ in } \mathbb{N} \}\).\qedhere
  \end{enumerate}
\end{exercise}

\begin{enumerate}
  \item $\forall x\ x+a=x$.
  \item $\forall x\ x\cdot a=x$.
  \item $\exists y\forall x(x\cdot y=x\wedge n=m+y)$.
  \item $\exists y\forall x\exists k(x+y=x\wedge \neg k=y\wedge n=m+k)$.
\end{enumerate}

\begin{exercise}{11.5}
  [Enderton, ex.~6, p.~146]
  Let $\Sigma_1$ and $\Sigma_2$ be sets of sentences such that nothing is a model of both $\Sigma_1$ and $\Sigma_2$. Show that there is a sentence $\tau$ such that
  \[
    \text{Mod } \Sigma_1 \subseteq \text{Mod } \tau \quad \text{and} \quad \text{Mod } \Sigma_2 \subseteq \text{Mod } \lnot \tau.
  \]
  (This can be stated: Disjoint $\text{EC}_\Delta$ classes can be separated by an EC class.) \textit{Suggestion}: $\Sigma_1 \cup \Sigma_2$ is unsatisfiable; apply compactness.
\end{exercise}

We may suppose $\Sigma_1$ and $\Sigma_2$ are satisfiable (the other cases are ommitted as trivial). $\Sigma_1\cup \Sigma_2$ is not satisfiable, thus not finitely satisfiable. Say that a finite subset $\Sigma_0$ is inconsistent. Let $\alpha$ be the conjunction of members of $\Sigma_0\cap \Sigma_1$. Clearly $\Sigma_1\vdash \alpha$. We also have that $\Sigma_2\vdash \neg \alpha$, for that if $\Sigma_2;\alpha$ is satisfiable so would be $\Sigma_0$ and thus contradiction, then we are done.
